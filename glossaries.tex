%!TEX root=../thesis.tex
% Glossar

% Glossar von: Sebastian Pollak

\newglossaryentry{react}{
    name={React},
    description={Eine JavaScript-Bibliothek zur Erstellung von Benutzeroberflächen, die Komponenten und Hooks nutzt, um wiederverwendbaren Code zu fördern.}
}
\newglossaryentry{usability}{
	name={Benutzerfreundlichkeit},
	description={Grad der Effizienz, Effektivität und Zufriedenheit, mit dem Nutzer eine Anwendung verwenden können.}
}
\newglossaryentry{data-integrity}{
	name={Datenintegrität},
	description={Sicherstellung, dass Daten korrekt, vollständig und unverändert bleiben.}
}

\newglossaryentry{components}{
name={Komponenten},
description={Wiederverwendbare Codebausteine in React und Next.js, die UI-Elemente und Funktionalitäten kapseln.}
}

\newglossaryentry{routing}{
name={Routing},
description={In Next.js: Automatische Zuordnung von URL-Pfaden zu Seiten basierend auf der Verzeichnisstruktur im app-Ordner.}
}

\newglossaryentry{jsx}{
name={JSX},
description={Eine Syntaxerweiterung für JavaScript, die es ermöglicht, HTML-ähnlichen Code in JavaScript-Dateien zu schreiben.}
}

\newglossaryentry{state-management}{
name={State-Management},
description={Verwaltung und Aktualisierung des Zustands einer Anwendung, oft unter Verwendung von React Hooks wie useState.}
}

\newglossaryentry{form-validation}{
name={Formularvalidierung},
description={Überprüfung der Benutzereingaben in einem Formular auf Korrektheit und Vollständigkeit vor der Verarbeitung oder Übermittlung der Daten.}
}

\newglossaryentry{nextjs}{
name={Next.js},
description={Ein React-Framework für die Erstellung von Webanwendungen mit serverseitigem Rendering und Routing.}
}

% Glossar von: Sebastian Sailer

\newglossaryentry{Persistenz}{
	name={persistente},
	description={\enquote{In der Informatik bezeichnet Persistenz die dauerhafte Speicherung von Daten, sodass diese auch in Zukunft oder nach einem Neustart des Systems wiederverwendet werden können. Dabei ist es entscheidend, dass die Daten zuverlässig und über längere Zeiträume hinweg erhalten bleiben und jederzeit zugänglich sind.} \cite{ChatGPT:PersistenzFormulierung}\cite{wiki:Persistenz}}
}

\newglossaryentry{Redundanz}{
	name={Redundanz},
	description={\enquote{Redundanz bezeichnet Daten, die mehrfach in einer Informationsquelle vorhanden sind und ohne Informationsverlust weggelassen werden können.}
    \cite{ChatGPT:RedundanzFormulierung}\cite{wiki:Redundanz}}}

\newglossaryentry{CRUD}{
	name={CRUD},
	description={\enquote{CRUD-Operationen sind die vier wichtigsten Funktionen für den Betrieb eines DBMS: Erstellen (Create), Lesen (Read), Aktualisieren (Update) und Löschen (Delete).}
    \cite{wiki:CRUD}}}

\newglossaryentry{Paradigmenwechsel}{
	name={Paradigmenwechsel},
	description={\enquote{Wechsel von einer wissenschaftlichen Grundauffassung zu einer anderen.}
    \cite{Definition:Paradigmenwechsel}}}
    
\newglossaryentry{System R}{
	name={System R},
	description={\enquote{System R war ein Forschungsprototyp von IBM in den 1970er Jahren und das erste relationale Datenbankmanagementsystem, das die Machbarkeit und Leistungsfähigkeit des relationalen Modells von E. F. Codd demonstrierte. Es legte den Grundstein für SQL}
    \cite{Paper:Geschichte1}}}


\newglossaryentry{SQL}{
	name={SQL},
	description={\enquote{SQL (ursprünglich SEQUEL) wurde für das \gls{System R} entwickelt. Es ist eine relationale Abfragesprache, die auf englischen Wörtern wie „select“ und „from“ basiert. Seit der Standardisierung durch ANSI und ISO ab 1986 ist SQL die führende Sprache für Datenbankabfragen und -interaktionen.}
    \cite{Buch:EdwinSchicker} S.91, \cite{Buch:AndreasMaier} S.105}}

    

\newglossaryentry{Kosten}{
	name={Kosten},
	description={\enquote{And because of the liberal license, PostgreSQL can be used, modified, and distributed by anyone free of charge for any purpose, be it private, commercial, or academic.}-\textit{PostgreSQL Dokumentation S.32 \cite{PostgreSQL:Hersteller}}}
}

\newglossaryentry{Open-Source}{
	name={Open-Source},
	description={\enquote{Als Open Source [...] wird Software bezeichnet, deren Quelltext öffentlich ist und von Dritten eingesehen, geändert und genutzt werden kann}-\cite{wiki:Open-Source}}
}

\newglossaryentry{Fork}{
	name={Fork},
	description={\enquote{Eine Abspaltung [...] ist in der Softwareentwicklung ein Entwicklungszweig nach der Aufspaltung eines Projektes in zwei oder mehrere Folgeprojekte}-\cite{wiki:Fork}}
}

\newglossaryentry{Datenbank-Sharding}{
	name={Datenbank-Sharding},
	description={ Hierbei wird eine große Datenbank über mehr als einen Computer hinweg gespeichert. Daten werden in mehrere \textbf{Shards} aufgeteilt und auf \textbf{mehreren} Datenbankservern gespeichert -\cite{aws:database-sharding}}
}

\newglossaryentry{Amazon Web Services}{
	name={Amazon Web Services}, 
	description={ ist ein US-amerikanischer Anbieter für Cloud-Computing-Dienste. Das Unternehmen wurde 2006 als Tochtergesellschaft von Amazon gegründet, mit dem Ziel, Entwicklern eine flexible IT-Infrastruktur bereitzustellen, die nach Bedarf genutzt werden kann.-\cite{wiki:aws}}
}

\newglossaryentry{Amazon S3}{
	name={Amazon S3}, 
	description={ ist ein skalierbarer und sicherer Objektspeicher-Service, der Daten für vielfältige Anwendungsfälle wie Archivierung, Sicherung und Apps speichert. Es bietet kosteneffiziente Speicherklassen, einfache Verwaltung und anpassbare Zugriffskontrollen. Amazon S3 steht hierbei für \textbf{Amazon Simple Storage Service} -\cite{Amazon:S3}}
}

\newglossaryentry{3. Normalform}{
	name={3. Normalform}, 
	description={ Normalisierung ist ein Entwurfsansatz für relationale Datenbanken mit dem Zweck, redundante Speicherung von Informationen und damit Inkonsistenz und Anomalien zu vermeiden \textit{Wikipedia, Normalisierung (Datenbank)} -\cite{wiki:normalformen}}
}

\newglossaryentry{Docker}{
	name={Docker}, 
	description={ "Docker ist eine freie Software zur Isolierung von Anwendungen mit Hilfe von Containervirtualisierung."-\cite{wiki:docker} }
}



\newacronym{ac-CRUD}{CRUD}{Create-Read-Update-Delete}
\newacronym{ac-SQL}{SQL}{Structured Query Language}
\newacronym{ac-DBMS}{DBMS}{Datenbankmanagementsystem}
\newacronym{ac-ORDB}{ORDB}{objektrelationales Datenbankmanagementsystem}
\newacronym{ac-JSON}{JSON}{JavaScript Object Notation}
\newacronym{ac-AWS}{AWS}{Amazon Web Services}
\newacronym{ac-S3}{Amazon S3}{Amazon Simple Storage Service}
\newacronym{ac-erd}{ER-Diagramm}{Entity-Relationship-Diagram}
 

% Glossar von: Manuel Fellner

\newglossaryentry{deployment}{
    name = {deployment},
    first={Deployment, Softwareverteilung},
    description={\enquote{Softwareverteilung (englisch software deployment) nennt man Prozesse zur Installation von Software auf Rechnern. Viele Anwender verfügen nicht über die Kenntnisse oder Berechtigungen, um Software selbst zu installieren. Daher ist es in Organisationen üblich, dass qualifizierte Mitarbeiter diese Aufgabe für die Anwender erledigen bzw. veranlassen. In größeren Organisationen wird Software unbeaufsichtigt installiert. Beispiel dafür sind die unbeaufsichtigte Installation unter Windows und die unbeaufsichtigte Installation unter Linux.} \cite{wiki-deployment}}
}

\newglossaryentry{api}{
    name={api},
    first={Application Programming Interface},
    description={\enquote{Eine Programmierschnittstelle (auch Anwendungsschnittstelle, genauer Schnittstelle zur Programmierung von Anwendungen), häufig nur kurz API genannt (von englisch application programming interface, wörtlich ‚Anwendungsprogrammierschnittstelle‘), ist ein Programmteil, der von einem Softwaresystem anderen Programmen zur Anbindung an das System zur Verfügung gestellt wird.} \cite{wiki-api}}
}

\newglossaryentry{interface}{
    name={interface},
    first={interface},
    description={\enquote{An interface is a well-defined entry point into a system.} (API development 2018, S. 1; Sascha Preibisch)\cite{book-api-development-springer}, auf Deutsch: \enquote{Eine Schnittstelle ist ein klar definierter Einstiegspunkt in ein System.} (Übersetzt durch \cite{deepl-trans-interface})}
}

\newglossaryentry{homogen}{
    name={homogen},
    first={Homogenität},
    description={\enquote{Homogenität bezeichnet die Gleichheit einer Eigenschaft, über die gesamte Ausdehnung eines Systems oder auch die Gleichartigkeit von Elementen eines Systems.} \cite{wiki-homogen}}
}

\newglossaryentry{rest}{
    name={REST},
    first={Representational State Transfer},
    description={\enquote{REST ist eine Softwarearchitektur, welche insbesondere für Webservices verwendet wird. REST ist eine Abstraktion der Struktur und des Verhaltens des World Wide Web. REST hat das Ziel, einen Architekturstil zu schaffen, der den Anforderungen des modernen Web besser genügt. Dabei unterscheidet sich REST vor allem in der Forderung nach einer einheitlichen Schnittstelle (siehe Abschnitt Prinzipien) von anderen Architekturstilen. } \cite{wiki-rest}}
}

\newglossaryentry{soap}{
    name={SOAP},
    first={Simple Object Access Protocol},
    description={\enquote{SOAP ist ein altes Netzwerkprotokoll, mit dessen Hilfe Daten zwischen Systemen ausgetauscht und RPCs (\gls{rpc}) durchgeführt werden können.} \cite{wiki-soap}}
}

\newglossaryentry{rpc}{
    name={RPC},
    first={Remote Procedure Call},
    description={\enquote{Remote Procedure Call (RPC; englisch für „Aufruf einer fernen Prozedur“) ist eine Technik zur Realisierung von Interprozesskommunikation. Sie ermöglicht den Aufruf von Funktionen in anderen Adressräumen. Im Normalfall werden die aufgerufenen Funktionen auf einem anderen Computer als das aufrufende Programm ausgeführt. Es existieren viele Implementierungen dieser Technik, die in der Regel untereinander nicht kompatibel sind. } \cite{wiki-rpc}}
}

\newglossaryentry{xml}{
    name={XML},
    first={Extensible Markup Language},
    description={\enquote{Remote Procedure Call (RPC; englisch für „Aufruf einer fernen Prozedur“) ist eine Technik zur Realisierung von Interprozesskommunikation. Sie ermöglicht den Aufruf von Funktionen in anderen Adressräumen. Im Normalfall werden die aufgerufenen Funktionen auf einem anderen Computer als das aufrufende Programm ausgeführt.} \cite{wiki-rpc}}
}

\newglossaryentry{req-body}{
    name={Request Body},
    first={Request Body},
    description={\enquote{Der Request Body ist der Bereich in einer Anfrage, in welchem sich die Daten befinden, die an den Server gesendet werden.} \cite{website-http-req-body}}
}

\newglossaryentry{idempotent}{
    name={Idempotent},
    first={Idempotent},
    description={\enquote{Eine HTTP-Methode ist idempotent, wenn die beabsichtigte Wirkung auf den Server bei einer einzelnen Anfrage diesselbe ist wie die Wirkung bei mehreren identischen Anfragen.} \cite{website-http-idempotent}}
}

\newglossaryentry{url}{
    name={URL},
    first={Uniform Resource Locator},
    description={\enquote{Ein Uniform Resource Locator (URL) ist eine Textzeichenfolge, die angibt, wo eine Ressource (wie eine Webseite, ein Bild oder ein Video) im Internet zu finden ist.} \cite{website-web-url}}
}

\newglossaryentry{restful-app}{
    name={RESTful Applikationen},
    first={RESTful Applikationen},
    description={\enquote{Eine Applikation wird dann RESTful genannt, wenn sie den 5 REST-Prinzipien folgt (diese Architektur also implementiert hat).}}
}

\newglossaryentry{endpoints}{
    name={Endpoints, Einstiegspunkte},
    first={Endpoints, Einstiegspunkte},
    description={\enquote{Ein (API-) Endpunkt (im Deutschen auch Einstiegspunkt genannt), ist ein Punkt in einer Applikation, der von einem Client abgefragt werden kann, um einen gewissen Zweck zu erzielen.}}
}

\newglossaryentry{hateoas}{
    name={Hypermedia as the Engine of Application State},
    first={Hypermedia as the Engine of Application State},
    description={\enquote{Hypermedia as the Engine of Application State (zu Deutsch: Hypermedia als Motor des Anwendungs-Zustands), ist ein Grundlegendes Prinzip der REST-Anwendungsarchitektur. Dies beschreibt, dass ein Client nur sehr wenig Ahnung über die REST-Schnittstellen eines Servers haben muss, sich aber trotzdem gute Informationen beschaffen kann. Hierfür braucht der Client nur das Basiswissen von Hypermedien. } \cite{wiki-hateoas}}
}

\newglossaryentry{uuid}{
    name={UUID},
    first={Universally Unique Identifier},
    description={\enquote{Ein Universally Unique Identifier (UUID) ist eine 128-Bit Zahl welche dazu dient, Objekte eindeutig zu identifizieren. Zum Beispiel: \textsc{550e8400-e29b-11d4-a716-446655440000} } \cite{wiki-uuid}}
}

\newglossaryentry{ddos}{
    name={DDOS},
    first={Distributed Denial of Service},
    description={\enquote{Ein Distributed Denial of Service (DDOS) ist ein Angriff, welcher das Ziel hat, einen Server oder ein Netzwerk zu überlasten. Dies geschieht durch das konstante Anfragen verschiedenster ressourcen des Servers mit vielen Rechnern. } \cite{wiki-ddos}}
}

\newglossaryentry{tls-ssl}{
    name={TLS/SSL},
    first={Transport Layer Security/Secure Socket Layer},
    description={\enquote{Transport Layer Security (TLS), auch bekannt unter der Vorgängerbezeichnung Secure Sockets Layer (SSL), ist ein Verschlüsselungsprotokoll zur sicheren Datenübertragung im Internet. } \cite{wiki-tls-ssl}}
}

\newglossaryentry{sql}{
    name={SQL},
    first={Structured Query Language},
    description={\enquote{SQL ist eine Datenbanksprache zur Definition von Datenstrukturen in relationalen Datenbanken sowie zum Bearbeiten (Einfügen, Verändern, Löschen) und Abfragen von darauf basierenden Datenbeständen. } \cite{wiki-sql}}
}

\newglossaryentry{full-stack-framework}{
    name={Full-Stack Framework},
    description={\enquote{Ein Full-Stack Framework bietet, im Vergleich zu dedizierten Frontend- oder Backend-Frameworks, die Möglichkeit, das Front- sowie das Backend mit einer Technologie einheitlich zu programmieren. Hier ist es nicht notwendig, mit eigenen API-Schnittstellen zu arbeiten, da alle Komponenten der Applikation voll integriert sind.}}
}

\newglossaryentry{sql-injection}{
    name={SQL-Injection},
    description={\enquote{Eine SQL-Injection ist ein Angriff, bei dem Schwachstellen in der Programmierung einer Anwendung ausgenutzt werden, die mit einer SQL-Datenbank kommuniziert. Solche Schwachstellen entstehen durch Fehler im Umgang mit Benutzereingaben, die nicht korrekt geprüft oder gefiltert werden. Dadurch können Angreifer schädliche SQL-Befehle einschleusen, um beispielsweise unerlaubt Daten auszulesen, Daten zu manipulieren oder zu löschen und in manchen Fällen sogar die Kontrolle über den gesamten Datenbankserver zu übernehmen. } \cite{wiki-sql-injection}}
}

\newglossaryentry{xss-attack}{
    name={XSS-Angriff},
    first={Cross-Site-Scripting},
    description={\enquote{Cross-Site-Scripting (XSS) ist eine Angriffsmethode, bei der Schwachstellen in Webanwendungen ausgenutzt werden, um schädlichen Code in den Browser eines Benutzers einzuschleusen. Der schädliche Code wird innerhalb des vertrauenswürdigen Kontexts der Webanwendung ausgeführt, oft ohne dass der Benutzer dies bemerkt. Dadurch können Angreifer sensible Daten stehlen, Benutzersitzungen übernehmen oder Schadcode ausführen, der beispielsweise zum Identitätsdiebstahl oder zur Kontrolle von Benutzerkonten genutzt wird. } \cite{wiki-xss-attack}}
}

\newglossaryentry{caching}{
    name={Caching},
    description={\enquote{Caching bezeichnet einen schnellen Pufferspeicher, der häufige Zugriffe auf langsame Datenspeicher oder rechenintensive Prozesse optimiert. Bereits geladene oder generierte Daten werden im Cache gespeichert, sodass sie bei späterem Bedarf schneller verfügbar sind. Zudem können voraussichtlich benötigte Daten vorab geladen und bereitgestellt werden (read-ahead). Caches können sowohl als Hardware (z. B. Speicherchips) als auch als Software (z. B. temporäre Dateien) realisiert sein. } \cite{wiki-cache}}
}

\newglossaryentry{api-first-method}{
    name={API-first Architektur},
    description={\enquote{Ein Entwicklungsansatz, bei dem die Definition und Gestaltung von Application Programming Interfaces (APIs) im Mittelpunkt stehen. Bevor andere Komponenten einer Anwendung entwickelt werden, werden die APIs spezifiziert, um sicherzustellen, dass alle Dienste und Anwendungen nahtlos miteinander kommunizieren können. Dies fördert eine konsistente und effiziente Integration verschiedener Systeme und ermöglicht eine parallele Entwicklung von Frontend- und Backend-Komponenten. }\cite{website-api-first-architektur}}
}

\newglossaryentry{service-orientated-architecture}{
    name={SOA},
    first={Service-orentierte Architektur},
    description={\enquote{Ein Architekturansatz, bei dem Anwendungen in unabhängige, lose gekoppelte Dienste zerlegt werden. Jeder Service erfüllt eine spezifische Aufgabe und kommuniziert über Schnittstellen wie RESTful APIs. Spring Boot vereinfacht die Umsetzung von SOA durch integrierte Tools wie Spring Cloud, Eureka (Service Discovery) und Zuul (API Gateway). Die Architektur bietet Vorteile wie Skalierbarkeit, Flexibilität und Wartbarkeit, indem Services unabhängig entwickelt, bereitgestellt und skaliert werden können. } \cite{wiki-service-orientated}}
}

\newglossaryentry{json}{
    name={JSON},
    first={JavaScript Object Notation},
    description={\enquote{JSON (JavaScript Object Notation) ist ein schlankes Datenaustauschformat, das für Menschen einfach zu lesen und zu schreiben und für Maschinen einfach zu parsen (Analysieren von Datenstrukturen) und zu generieren ist. Es basiert auf einer Untermenge der JavaScript Programmiersprache, Standard ECMA-262 dritte Edition - Dezember 1999. } \cite{website-json-definition}}
}


\newglossaryentry{jpa}{
    name={JPA},
    first={Jakarta Persistence API},
    description={\enquote{Jakarta Persistence API (JPA) ist eine Spezifikation für das objekt-relationale Mapping (ORM) in Jakarta, die es ermöglicht, relationale Datenbanken über Jakarta-Objekte zu verwalten. Sie abstrahiert SQL-Operationen und vereinfacht den Datenbankzugriff durch Annotationen und automatisches Mapping von Entitäten. JPA wird oft mit Implementierungen wie Hibernate oder EclipseLink verwendet und ist ein zentraler Bestandteil moderner Jakarta-Anwendungen, insbesondere in Verbindung mit Spring Boot und Spring Data JPA.} \cite{website-jpa} \cite{prompt-gpt-jpa-summary}}
}

\newglossaryentry{crud}{
    name={CRUD},
    first={Create, Delete, Update, Delete},
    description={\enquote{CRUD steht für Create, Read, Update, Delete und beschreibt die vier grundlegenden Operationen zur Verwaltung von Daten in einer Datenbank oder einem System. Create fügt neue Daten hinzu, Read liest bestehende Daten aus, Update ändert vorhandene Einträge, und Delete entfernt Daten. Diese Funktionen sind essenziell für Anwendungen mit Datenbankanbindung und werden häufig über REST-APIs, SQL-Abfragen oder ORMs wie JPA umgesetzt.}\cite{prompt-gpt-crud-summary}}
}

\newglossaryentry{cors}{
    name={CORS},
    first={Cross-Origin Resource Sharing},
    description={\enquote{CORS ist ein Sicherheitsmechanismus, der von Webbrowsern verwendet wird, um kontrollierten Zugriff auf Ressourcen zwischen verschiedenen Ursprüngen (Domains) zu ermöglichen. Standardmäßig blockieren Browser aus Sicherheitsgründen Anfragen von einer Domain an eine andere (Same-Origin-Policy). CORS erlaubt jedoch explizite Freigaben, indem der Server in den HTTP-Headern angibt, welche Ursprünge, Methoden und Header erlaubt sind. Dies ist besonders wichtig für APIs, die von verschiedenen Webanwendungen genutzt werden.}\cite{prompt-gpt-cors-summary}}
}

\newglossaryentry{csrf}{
    name={CSRF},
    first={Cross-Site Requst Forgery},
    description={\enquote{CSRF ist eine Art von Angriff, bei dem ein Angreifer einen authentifizierten Benutzer dazu bringt, unbeabsichtigt schädliche Anfragen an eine vertrauenswürdige Website zu senden. Dabei wird die Sitzung des Benutzers missbraucht, um Aktionen in seinem Namen auszuführen, z. B. das Ändern von Passwörtern oder das Tätigen von Überweisungen.
Zum Schutz vor CSRF-Angriffen setzen moderne Webanwendungen auf Mechanismen wie CSRF-Tokens, die für jede Anfrage individuell generiert und überprüft werden, oder auf SameSite-Cookies, die die Übermittlung von Cookies auf fremden Websites einschränken.}\cite{prompt-gpt-csrf-summary} \cite{website-csrf-explanation}}
}

\newglossaryentry{jwt}{
    name={JWT},
    first={JSON Web Token},
    description={\enquote{ JWT ist ein offener Standard (RFC 7519) für die sichere Übertragung von Informationen zwischen zwei Parteien als JSON-Objekt. Es wird häufig für Authentifizierungs- und Autorisierungszwecke in Webanwendungen verwendet. Ein JWT besteht aus drei Teilen: Header, Payload und Signatur. Die Signatur sorgt dafür, dass das Token nicht manipuliert werden kann. JWTs werden oft in HTTP-Headern verwendet, um Benutzersitzungen ohne serverseitigen Speicher zu verwalten.}\cite{prompt-gpt-jwt-summary}}
}

\newglossaryentry{dependencyinjection}{
    name={DI},
    first={Dependency Injection},
    description={\enquote{ Dependency Injection ist ein Entwurfsmuster in der Softwareentwicklung, das die Abhängigkeiten einer Klasse von außen bereitstellt, anstatt sie intern zu instanziieren. Dadurch wird der Code flexibler, testbarer und besser wartbar. In Spring Boot wird DI automatisch durch den Spring IoC (Inversion of Control) Container gehandhabt, der Abhängigkeiten erkennt und verwaltet. Eine gängige Methode ist die Verwendung der @Autowired-Annotation, um Objekte automatisch zu injizieren.}\cite{prompt-gpt-dependency-injection-summary}}
}

% Glossar von: Momo

\newglossaryentry{mvc}{
    name={MVC},
    first={Model-View-Controller},
    description={\enquote{MVC (Model-View-Controller) ist ein Software-Designmuster, das in Frameworks verwendet wird, um die Codebasis in drei Komponenten zu trennen: Model (Datenlogik), View (Präsentation der Daten) und Controller (Interaktion zwischen Model und View). Diese Struktur erleichtert die Wiederverwendbarkeit und Wartung von Code. \cite{madurapperuma2022state}}}
}

\newglossaryentry{dom}{
    name={DOM},
    first={Document Object Model},
    description={\enquote{Das Document Object Model (DOM) ist eine Schnittstelle, die die Struktur eines HTML- oder XML-Dokuments als Baum darstellt. Es ermöglicht Frameworks wie React, Änderungen effizient zu verarbeiten, indem ein virtueller DOM genutzt wird, der Unterschiede erkennt und nur notwendige Updates durchführt. \cite{hutagikar2020analysis}}}
}

\newglossaryentry{virtual-dom}{
    name={Virtueller DOM},
    first={Virtueller DOM},
    description={\enquote{Der virtuelle DOM ist eine leichte Kopie des realen DOMs. Frameworks wie React verwenden ihn, um Änderungen effizient zu berechnen und nur aktualisierte Elemente im realen DOM zu verändern, was die Performance verbessert. \cite{rathinam2022analysis}}}
}

\newglossaryentry{mvvm}{
    name={MVVM},
    first={Model-View-ViewModel},
    description={\enquote{Das MVVM (Model-View-ViewModel)-Muster wird häufig in Frameworks wie Vue.js genutzt. Es trennt die Datenlogik (Model), die Präsentation (View) und die Bindungslogik (ViewModel), was eine bessere Trennung von Verantwortlichkeiten ermöglicht. \cite{awasthiresearch}}}
}

\newglossaryentry{typescript}{
    name={TypeScript},
    first={TypeScript},
    description={\enquote{TypeScript ist eine Programmiersprache, die auf JavaScript basiert und in Frameworks wie Angular verwendet wird. Sie bietet zusätzliche Features wie statische Typisierung, die zur Fehlervermeidung und besseren Wartbarkeit beiträgt. \cite{shetty2020review}}}
}

\newglossaryentry{data-binding}{
    name={Datenbindung},
    first={Datenbindung},
    description={\enquote{Datenbindung bezieht sich auf die automatische Synchronisation zwischen der Benutzeroberfläche (View) und der Datenquelle (Model). Frameworks wie Angular und Vue.js unterstützen sowohl Einweg- als auch Zweiweg-Datenbindung. \cite{hutagikar2020analysis}}}
}

\newglossaryentry{spa}{
    name={SPA},
    first={Single-Page Application (SPA)},
    description={\enquote{Eine Single-Page Application (SPA) ist eine Webanwendung, bei der die Inhalte dynamisch geladen und aktualisiert werden, ohne die gesamte Seite neu zu laden. Frameworks wie React und Angular werden häufig für SPAs verwendet. \cite{awasthiresearch}}}
}

\newglossaryentry{scalability}{
    name={Skalierbarkeit},
    first={Skalierbarkeit},
    description={\enquote{Skalierbarkeit beschreibt die Fähigkeit eines Systems, mit wachsenden Anforderungen effizient zu wachsen. Frameworks wie Node.js und Angular sind für skalierbare Anwendungen konzipiert. \cite{madurapperuma2022state}}}
}

\newglossaryentry{jsx}{
    name={JSX},
    first={JavaScript XML (JSX)},
    description={\enquote{JSX ist eine Erweiterung von JavaScript, die es Entwicklern ermöglicht, HTML-ähnlichen Code in JavaScript zu schreiben. Sie wird in React verwendet, um Benutzeroberflächen-Komponenten zu definieren.} \cite{rathinam2022analysis}}}

\newglossaryentry{cli}{
    name={Command Line Interface (CLI)},
    description={Ein Werkzeug, das die Befehlszeilensteuerung für Entwicklungsprozesse ermöglicht. Angular CLI wird z. B. zur Initialisierung und Verwaltung von Projekten genutzt. \cite{angular_blog}}
}

\newglossaryentry{reaktivitaetssystem}{
    name={Reaktivitätssystem},
    description={Ein System, das Änderungen im Datenmodell erkennt und die Benutzeroberfläche automatisch aktualisiert. Es wird in Frameworks wie Vue.js genutzt. \cite{vue_blog}}
}


\newglossaryentry{seo}{
    name={SEO},
    first={Search Engine Optimization (SEO)},
    description={\enquote{Eine Reihe von Strategien und Techniken, die darauf abzielen, die Sichtbarkeit und Platzierung einer Website in den Suchmaschinenergebnissen zu verbessern. Durch serverseitiges Rendering und statische Generierung, wie sie von Next.js unterstützt werden, kann die SEO-Performance erheblich gesteigert werden. \cite{shetty2020review}}}
}

\newglossaryentry{server-rendering}{
    name={Server-Rendering},
    first={Serverseitiges Rendering (Server-Rendering)},
    description={\enquote{Eine Technik, bei der Webseiten auf einem Server gerendert und als vollständig generierte HTML-Seiten an den Client gesendet werden. Dies verbessert die SEO und reduziert die Ladezeiten. \cite{shetty2020review}}}
}

\newglossaryentry{static-generation}{
    name={Statische Generierung},
    first={Statische Generierung},
    description={\enquote{Ein Prozess, bei dem HTML-Dateien während der Build-Zeit generiert werden. Diese Dateien werden dann an den Client ausgeliefert, was schnelle Ladezeiten und eine verbesserte Performance gewährleistet. \cite{shetty2020review}}}
}

\newglossaryentry{code-splitting}{
    name={Code-Splitting},
    first={Code-Splitting},
    description={\enquote{Eine Technik, bei der der JavaScript-Code einer Anwendung in kleinere Teile aufgeteilt wird. Diese Teile werden nur geladen, wenn sie benötigt werden, um die Performance zu optimieren. \cite{nextjs_blog}}}
}

\newglossaryentry{image-optimization}{
    name={Bildoptimierung},
    first={Bildoptimierung},
    description={\enquote{Ein Verfahren, um Bilder in einer Anwendung effizient zu laden. Es verbessert die Ladezeiten und die Benutzererfahrung, indem Bilder automatisch in der richtigen Größe und im passenden Format bereitgestellt werden. \cite{rathinam2022analysis}}}
}

\newglossaryentry{automatic-routing}{
    name={Automatisches Routing},
    first={Automatisches Routing},
    description={\enquote{Eine Funktion, bei der URLs und Routen in einer Anwendung automatisch basierend auf der Ordnerstruktur erstellt werden, ohne dass eine manuelle Konfiguration erforderlich ist. \cite{shetty2020review}}}
}

\newglossaryentry{hot-reloading}{
    name={Hot-Reloading},
    first={Hot-Reloading},
    description={\enquote{Eine Funktion, die es Entwicklern ermöglicht, Änderungen an ihrem Code sofort in der Anwendung zu sehen, ohne die Seite neu laden zu müssen. Dies beschleunigt die Entwicklungsprozesse. \cite{madurapperuma2022state}}}
}

\newglossaryentry{architekturdesign}{
    name={Architekturdesign},
    first={Architekturdesign},
    description={\enquote{Beschreibung der Struktur einer Anwendung, die Komponenten und deren Interaktionen definiert, um eine modulare und wartbare Entwicklung zu ermöglichen.}}
}

\newglossaryentry{rest-api}{
    name={REST-API},
    first={REST-API},
    description={\enquote{Eine Schnittstelle für die Kommunikation zwischen Frontend und Backend, die HTTP-Protokolle nutzt, um Daten zu übertragen. REST steht für Representational State Transfer. \cite{shetty2020review}}}
}

\newglossaryentry{spring-boot}{
    name={Spring Boot},
    first={Spring Boot},
    description={\enquote{Ein Java-basiertes Framework, das die Entwicklung von Microservices und serverseitigen Anwendungen erleichtert. Es bietet Tools für die Erstellung und Verwaltung von REST-APIs. \cite{shetty2020review}}}
}

\newglossaryentry{mockups}{
    name={Mockups},
    first={Mockups},
    description={\enquote{Statische Darstellungen eines Designs, die verwendet werden, um das Layout und die visuelle Gestaltung einer Anwendung zu zeigen und Feedback einzuholen. \cite{ux_usability}}}
}

\newglossaryentry{interaktive-prototypen}{
    name={Interaktive Prototypen},
    first={Interaktive Prototypen},
    description={\enquote{Simulationen einer Anwendung mit funktionalen Benutzerinteraktionen, die zur Durchführung von Usability-Tests verwendet werden. \cite{ux_usability}}}
}

\newglossaryentry{widget-basierte-darstellung}{
    name={Widget-basierte Darstellung},
    first={Widget-basierte Darstellung},
    description={\enquote{Ein Designansatz, bei dem Informationen in kleinen, anpassbaren Modulen (Widgets) organisiert werden, die vom Benutzer angepasst oder verschoben werden können.}}
}

\newglossaryentry{responsive-design}{
    name={Responsive Design},
    first={Responsive Design},
    description={\enquote{Eine Technik, die sicherstellt, dass die Benutzeroberfläche auf verschiedenen Bildschirmgrößen gut aussieht und funktioniert, z. B. auf Smartphones, Tablets oder Desktops. \cite{shetty2020review}}}
}

\newglossaryentry{usability-tests}{
    name={Usability-Tests},
    first={Usability-Tests},
    description={\enquote{Verfahren zur Bewertung der Benutzerfreundlichkeit einer Anwendung, bei denen Benutzer typische Aufgaben ausführen, um Probleme zu identifizieren. \cite{ux_usability}}}
}

\newglossaryentry{personas}{
    name={Personas},
    first={Personas},
    description={\enquote{Fiktive Charaktere, die typische Benutzer einer Anwendung repräsentieren, um deren Bedürfnisse, Ziele und Herausforderungen zu verdeutlichen. \cite{ux_usability}}}
}

\newglossaryentry{progressive-offenlegung}{
    name={Progressive Offenlegung},
    first={Progressive Offenlegung},
    description={\enquote{Ein Prinzip im UX-Design, bei dem zusätzliche oder erweiterte Funktionen nur dann angezeigt werden, wenn der Benutzer sie benötigt.}}
}

