%!TEX root=../main.tex
\chapter{Einleitung} 

\section{Projektumfang}

Dieses Diplomprojekt wurde in Kooperation mit der Kurzzeitvermietungsfirma Equilibria GmbH realisiert und bildet einen wesentlichen Bestandteil der Digitalisierungsstrategie des Unternehmens. Ziel ist es, die bislang manuelle und fehleranfällige Verwaltung von Mieterdaten durch eine zentralisierte, digitale Lösung zu ersetzen.

Die bisherige Kommunikation mit Mietern fand überwiegend über Messenger-Dienste statt, was die Datenverarbeitung sowie das Auffinden von Informationen erschwerte. Das geplante System soll diesen Prozess durch ein Online-Formular mit Validierungs- und Speicherlogik sowie einer strukturierten Oberfläche zur Buchungsübersicht deutlich vereinfachen und effizienter gestalten.

\section{Ziele der Anwendung}

Das Hauptziel des Projekts ist die Entwicklung einer Webanwendung, die:

\begin{itemize}
    \item die Eingabe und Verwaltung von Mieterdaten erleichtert,
    \item Eingaben automatisiert auf Richtigkeit prüft,
    \item Informationen dauerhaft speichert,
    \item und auf Wunsch automatisch Buchungsprotokolle erstellt.
\end{itemize}

Darüber hinaus soll das System eine übersichtliche Benutzeroberfläche bieten, um getätigte Buchungen nachvollziehbar und einfach zugänglich darzustellen. Die langfristige Vision besteht darin, interne Verwaltungsprozesse des Unternehmens zu standardisieren, zu beschleunigen und für neue Standorte oder Mietmodelle skalierbar zu gestalten.

\section{Aufgabenstellung und Teilziele}

Zur Umsetzung dieses Ziels wurden folgende zentrale Aufgaben bzw. Fragestellungen definiert:

\begin{enumerate}
    \item Wie können Formulardaten im Hintergrund einer Webanwendung so verarbeitet werden, dass der Zugriff und die Verwaltung sowohl effizient als auch sicher gestaltet sind?
    \item Wie können Buchungsdaten so gespeichert und zugänglich gemacht werden, dass der Zugriff sowohl benutzerfreundlich als auch performant ist?
    \item Wie kann eine Mitarbeiteransicht gestaltet werden, die durch eine intuitive Benutzeroberfläche sowie effiziente und sichere Schnittstellen die Verwaltung von Buchungsdaten optimiert?
    \item Wie kann die Erfassung und Verarbeitung von Formulardaten in einer Webanwendung gestaltet werden, sodass die Eingabe effizient, sicher und benutzerfreundlich gewährleistet wird?
\end{enumerate}

\subsection{Backend und Datenhaltung}

Ein zentrales Ziel des Projekts ist die Entwicklung einer Backend-Schnittstelle, welche die sichere Verarbeitung, Validierung und langfristige Speicherung der erfassten Mieterdaten ermöglicht. Dabei soll eine relationale Datenbank in Kombination mit einer REST-konformen Programmierschnittstelle (API) zum Einsatz kommen, um eine skalierbare und modulare Systemarchitektur zu gewährleisten.

\subsection{Formularsystem und Mitarbeiteransicht}

Ein weiteres zentrales Ziel ist die Umsetzung eines Webformulars zur Erfassung der Mieterdaten, das die Eingaben automatisiert prüft und an das Backend übermittelt. Ergänzend dazu soll eine Mitarbeiteransicht entwickelt werden, welche den Mitarbeitenden der Equilibria GmbH eine strukturierte und intuitive Übersicht über Immobilien, Mieterinformationen und Buchungsprotokolle bietet. \cite{prompt-gpt-write-einleitung}
