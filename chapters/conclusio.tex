%!TEX root=../main.tex
\chapter{Conclusio} 

Im Rahmen dieser Diplomarbeit wurde ein webbasiertes System zur Erfassung, Verwaltung und Darstellung von Mieterdaten für die Equilibria GmbH konzipiert und prototypisch umgesetzt. Ziel war es, bestehende manuelle Prozesse – insbesondere die dezentrale Kommunikation über Messenger-Dienste – durch eine zentrale, benutzerfreundliche und skalierbare Softwarelösung zu ersetzen.

Die im Projekt entwickelten Komponenten erfüllen die zuvor definierten funktionalen Anforderungen: Ein Online-Formular ermöglicht die strukturierte Eingabe von Mieterdaten mit integrierter Validierungslogik. Diese Daten werden über eine REST-konforme API an eine relationale Datenbank übermittelt und dort langfristig gespeichert. Ergänzend wurde eine Mitarbeiteransicht geschaffen, welche Buchungsinformationen übersichtlich darstellt und die tägliche Arbeit der Mitarbeitenden vereinfacht.

Besondere Aufmerksamkeit galt der sauberen Trennung zwischen Frontend und Backend, der Modularität des Systems sowie der zukünftigen Erweiterbarkeit – etwa im Hinblick auf neue Standorte, zusätzliche Funktionen oder die Integration in andere Systeme. Auch Aspekte wie Datensicherheit, Benutzerfreundlichkeit und formale Datenkonsistenz wurden im Projekt berücksichtigt.

Trotz begrenzter Ressourcen konnte ein stabiler Prototyp realisiert werden, der zeigt, dass digitale Lösungen auch mit einem kleinen Team und unter realistischen Rahmenbedingungen erfolgreich umsetzbar sind. Die entwickelte Anwendung stellt somit nicht nur einen direkten Mehrwert für den Auftraggeber dar, sondern dient auch als Referenz für vergleichbare Digitalisierungsprojekte in kleinen Unternehmen.

Insgesamt zeigt das Projekt, wie praxisnahe Softwareentwicklung im schulischen Kontext sinnvoll umgesetzt werden kann – von der ersten Anforderungsanalyse über die technische Umsetzung bis hin zur schriftlichen Dokumentation im Rahmen dieser Arbeit.