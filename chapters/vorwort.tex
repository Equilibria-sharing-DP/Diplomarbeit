%!TEX root=../main.tex
\chapter{Vorwort}

Diese Diplomarbeit befasst sich mit der Entwicklung eines digitalen Systems zur effizienteren Erfassung und Verwaltung von Mieterdaten im Rahmen von Kurzzeitvermietungen. Das Projekt wurde in Zusammenarbeit mit der Equilibira GmbH realisiert und zielt darauf ab, bestehende manuelle Prozesse – insbesondere die Kommunikation über verschiedene Messenger-Dienste – durch eine zentrale, strukturierte Lösung zu ersetzen.

Die Kapitel \enquote{Literaturrecherche} sowie \enquote{State of the Art} bieten eine thematische Einführung und erläutern grundlegende Begriffe, Technologien und Zielsetzungen des Projekts. In der \enquote{Technischen Machbarkeitsstudie} werden sowohl technische als auch organisatorische Anforderungen erfasst und auf ihre Umsetzbarkeit geprüft, um frühzeitig mögliche Herausforderungen zu erkennen.

Im Kapitel \enquote{Konzept} wird das technische Konzept der Anwendung ausgearbeitet. Dies umfasst die Systemarchitektur, das Design der Datenbank, die Struktur der Programmierschnittstellen (APIs), das Authentifizierungsmodell sowie die Benutzeroberfläche. Dabei wird auch auf Designentscheidungen eingegangen, die auf Nutzerfreundlichkeit und Erweiterbarkeit abzielen.

Das Kapitel \enquote{Implementierung} beschreibt die konkrete Umsetzung des Systems. Alle wesentlichen Bestandteile – vom Backend und der Datenbank bis hin zur Client-seitigen Logik und dem responsiven Frontend – werden detailliert erläutert. Die Entwicklungsschritte, eingesetzten Frameworks und gewählten Technologien werden begründet und anhand von Beispielen nachvollziehbar gemacht.

Die abschließende \enquote{Retrospektive} beleuchtet die größten Herausforderungen während der Projektumsetzung, dokumentiert aufgetretene Probleme sowie die gefundenen Lösungen und zieht Schlüsse für zukünftige Projekte. In der \enquote{Conclusio} werden die wichtigsten Ergebnisse zusammengefasst, ein Fazit gezogen und ein Ausblick auf mögliche Weiterentwicklungen des Systems gegeben.

Mit dieser Arbeit möchten wir nicht nur den Anforderungen unseres Auftraggebers gerecht werden, sondern auch zeigen, wie ein praxisrelevantes Softwareprojekt unter realen Bedingungen geplant, umgesetzt und dokumentiert werden kann. \cite{prompt-gpt-write-vorwort}

\textbf{Hinweis zur Sprache:}

Aus Gründen der besseren Lesbarkeit wird in dieser Arbeit bei personenbezogenen Bezeichnungen die männliche Form verwendet. Sie bezieht sich jedoch stets auf alle Geschlechter gleichermaßen. Diese sprachliche Vereinfachung dient ausschließlich der Textflussoptimierung und impliziert keinerlei Wertung.