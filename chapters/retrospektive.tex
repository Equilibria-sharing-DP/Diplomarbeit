%!TEX root=../main.tex
\chapter{Retrospektive}

\section{Erreichte Ergebnisse}

Der zu Beginn des Projekts definierte Umfang konnte erfolgreich umgesetzt werden. Die geplante Webanwendung wurde funktionsfähig realisiert und bietet die Möglichkeit, Mieterdaten effizient zu erfassen, zu validieren, langfristig zu speichern und in strukturierter Form darzustellen. Darüber hinaus ermöglicht das System die Erstellung von Buchungsprotokollen sowie eine übersichtliche Darstellung aller bisherigen und zukünftigen Buchungen. Die definierten Anforderungen konnten zum Großteil im vorgesehenen Zeitrahmen erfüllt werden.

\section{Relevanz und Auswirkungen}

\subsection{Gesellschaftliche Relevanz}

Die Digitalisierung administrativer Prozesse gewinnt zunehmend an Bedeutung – insbesondere in kleinen und mittelständischen Unternehmen, die bislang oft mit manuellen oder inoffiziellen Kommunikationskanälen arbeiten. Durch unser Projekt wurde ein exemplarischer Beitrag zur digitalen Transformation in der Kurzzeitvermietung geleistet. Die Anwendung dient als Beispiel dafür, wie mit begrenzten Ressourcen ein praxisrelevantes IT-System umgesetzt werden kann, das echten Mehrwert für den operativen Alltag bietet.

\subsection{Wirtschaftliche und betriebliche Relevanz}

Für den Auftraggeber stellt die entwickelte Lösung einen wesentlichen Fortschritt dar. Die Arbeitsabläufe im Bereich der Mieterkommunikation, Datenerfassung und Buchungsübersicht konnten deutlich vereinfacht und beschleunigt werden. Durch die automatisierte Datenverarbeitung und die strukturierte Darstellung aller Buchungen wird nicht nur Zeit gespart, sondern auch die Fehleranfälligkeit reduziert. Darüber hinaus legt das Projekt die Grundlage für zukünftige Erweiterungen – etwa für mehrere Standorte oder zusätzliche Funktionalitäten.

\newpage

\section{Herausforderungen während der Umsetzung}

Während der Umsetzung des Projekts sind mehrere unerwartete Herausforderungen aufgetreten:

\begin{itemize}
    \item \textbf{Kommunikation im Team:} Trotz regelmäßiger Meetings und Kommunikation wurden Änderungen und Anforderungen an Schnittstellen, Datenstrukturen, etc. teilweise unzureichend kommuniziert, was besonders in der Implementierungsphase zu Missverständnissen führte. Die Frontend- und Backend Bereiche waren bis zur Mitte der Implementierungsphase komplett isoliert voneinander, was zu einem Mehraufwand geführt hat. Für zukünftige Projekte empfiehlt sich ein klar strukturierter Kommunikationsplan und regelmäßige, klar strukturierte Meetings.
    
    \item \textbf{Unterschätzung neuer Technologien:} Die Komplexität einiger Technologien - insbesondere im Frontend-Bereich - waren zu Projektbeginn für das Team noch weitgehend unbekannt. Dies führte stellenweise zu Verzögerungen, da notwendige Kenntnisse erst erarbeitet werden mussten. Hier hätten externe Experten oder gezieltere Vorab-Recherche geholfen.
\end{itemize}

\section{Verbesserungspotenzial}

Neben den bereits genannten Herausforderungen gibt es auch einige Verbesserungsvorschläge für zukünftige Iterationen des Projekts:

\begin{itemize}
    \item \textbf{Strukturiertere API-Dokumentation:} Die aktuell dokumentierten Schnittstellen sind funktional, jedoch fehlt es stellenweise an übersichtlichen Beispielen und klaren Response-Strukturen. Eine ausführlichere API-Dokumentation würde insbesondere externen Entwickler:innen den Einstieg erleichtern.
    
    \item \textbf{Beitragsrichtlinien und Git-Nutzung:} Während der Entwicklung arbeitete jedes Teammitglied an eigenen Branches. Für eine verbesserte Struktur wären interne Richtlinien sowie die Aufstellung spezifischer Userstories hilfreich gewesen.
    
    \item \textbf{Hosting und Deployment:} Das Projekt wurde bis jetzt noch nicht online bereitgestellt, was zu einer Verzögerung beim Testing führte. Dies hat die Folgen, dass das Projekt im dafür vorgesehenen Zeitraum nicht extern vom Projektteam oder vom Auftraggeber getestet werden konnte.
\end{itemize} \cite{prompt-gpt-write-retrospektive}

